\documentclass[10pt,a4paper]{article}
\usepackage{authblk}
% \usepackage{times,latexsym}
\usepackage[T1]{fontenc}
% \usepackage[]{../common/acl2021}
% \usepackage[]{../common/tacl2018v2}

% replacement for style file
\usepackage{natbib}
% \usepackage{hyperref}
% \usepackage{url}
% \renewcommand{\UrlFont}{\ttfamily\small}

\usepackage{times}
\usepackage{latexsym}

\usepackage{url}
\renewcommand{\UrlFont}{\ttfamily\small}
\usepackage[utf8]{inputenc}
\usepackage{xcolor}

% Totally tabular!
\usepackage{booktabs}

\usepackage{graphicx}

\usepackage{caption}
\usepackage{subcaption}

\usepackage{amssymb}
\usepackage{amsmath}

\usepackage{textcomp}

\DeclareMathOperator*{\argmax}{argmax}
\DeclareMathOperator*{\argmin}{argmin}
\DeclareMathOperator*{\softmax}{softmax}
\DeclareMathOperator*{\logsumexp}{logsumexp}
\DeclareMathOperator*{\hardmax}{hardmax}
\DeclareMathOperator*{\entropy}{H}
\DeclareMathOperator{\expectation}{\mathbb{E}}
\DeclareMathOperator{\prob}{P}

\usepackage{hyperref}
\usepackage{cleveref}
\def\sectionautorefname{Section}
\def\subsectionautorefname{Section}

% (From ACL style:)
% This is not strictly necessary, and may be commented out,
% but it will improve the layout of the manuscript,
% and will typically save some space.
\usepackage{microtype}

% TODO: get a better draft watermark than this lol
% \usepackage{draftwatermark}
% \SetWatermarkText{DRAFT}
% \SetWatermarkScale{1}

% fix up author superscripts
\newcommand*\samethanks[1][\value{footnote}]{\footnotemark[#1]}
\renewcommand*{\Affilfont}{\normalsize\normalfont}
\renewcommand*{\Authfont}{\bfseries}
\renewcommand\footnotemark{}
\makeatletter
\renewcommand\AB@authnote[1]{\rlap{\textsuperscript{\normalfont#1}}}
\renewcommand\Authsep{,~~\,}
\renewcommand\Authand{~~and~}
\renewcommand\Authands{,~~and~}
\makeatother
% end fix up author superscripts

% appendix autoref patch (\autoref subsections in appendix)
% https://tex.stackexchange.com/questions/149807/autoref-subsections-in-appendix
\usepackage{appendix}
\usepackage{etoolbox}
\makeatletter
\patchcmd{\hyper@makecurrent}{%
    \ifx\Hy@param\Hy@chapterstring
        \let\Hy@param\Hy@chapapp
    \fi
}{%
    \iftoggle{inappendix}{%true-branch
        % list the names of all sectioning counters here
        \@checkappendixparam{chapter}%
        \@checkappendixparam{section}%
        \@checkappendixparam{subsection}%
        \@checkappendixparam{subsubsection}%
        \@checkappendixparam{paragraph}%
        \@checkappendixparam{subparagraph}%
    }{}%
}{}{\errmessage{failed to patch}}

\newcommand*{\@checkappendixparam}[1]{%
    \def\@checkappendixparamtmp{#1}%
    \ifx\Hy@param\@checkappendixparamtmp
        \let\Hy@param\Hy@appendixstring
    \fi
}
\makeatletter

\newtoggle{inappendix}
\togglefalse{inappendix}

\apptocmd{\appendix}{\toggletrue{inappendix}}{}{\errmessage{failed to patch}}
\apptocmd{\subappendices}{\toggletrue{inappendix}}{}{\errmessage{failed to patch}}
% end appendix autoref patch

% fancy underline
% https://alexwlchan.net/2017/10/latex-underlines/
\usepackage{contour}
\usepackage[normalem]{ulem}

\renewcommand{\ULdepth}{1.8pt}
\contourlength{0.8pt}

\newcommand{\nuline}[1]{%
  \uline{\phantom{#1}}%
  \llap{\contour{white}{#1}}%
}
% end fancy underline


\usepackage{todonotes}  % temporary - use this for margin notes in bubbles
\usepackage[normalem]{ulem}  % for \sout
\newif\ifcomments
\commentstrue
% \commentsfalse

\ifcomments
    \newcommand*{\TODO}[1]{\textcolor{red}{[TODO: #1]}}
    \newcommand*{\tocite}[1]{(\textcolor{blue}{#1})}
    \newcommand*{\tocitep}[1]{(\textcolor{blue}{#1})}
    \newcommand*{\tocitet}[1]{\textcolor{blue}{#1}}
    
    \newcommand*{\julian}[1]{\textcolor{orange}{[JJM: #1]}}
    \newcommand*{\ian}[1]{\textcolor{olive}{[IFT: #1]}}
    \newcommand*{\jan}[1]{\textcolor{purple}{[JAB: #1]}}
    
    \definecolor{iansnotecolor}{RGB}{255, 218, 181}
    \newcommand*{\iansidenote}[1]{
        \todo[color=iansnotecolor, size=\footnotesize]{%
        [\textbf{Ian:}] #1}%
    }
    
    \newcommand*{\proposed}[1]{\textcolor{purple}{#1}}
    \newcommand*{\maybedelete}[1]{\textcolor{red}{\sout{#1}}}
    \newcommand{\punch}[0]{\includegraphics[width=.04\textwidth]{images/falconpunch.png}}
\else
    %%
    % Disable user commands for submission or length check
    \newcommand*{\TODO}[1]{}
    \newcommand*{\tocite}[1]{}
    \newcommand*{\tocitep}[1]{}
    \newcommand*{\tocitet}[1]{}
    
    \newcommand*{\julian}[1]{}
    \newcommand*{\ian}[1]{}
    \newcommand*{\jan}[1]{}
    \newcommand*{\iansidenote}[1]{}
    
    \newcommand*{\maybedelete}[1]{}
    \newcommand*{\proposed}[1]{#1}
\fi
\usepackage{enumitem}
% \usepackage[margin=1.2in]{geometry}

% \usepackage{csquotes}

% \usepackage{titling}
% \setlength{\droptitle}{-120pt}
% \predate{\vspace{-50pt}\begin{center}\large}
% \postdate{\par\end{center}\vspace{-10pt}}

\title{Proposal: An Adversarial Evaluation of Adversarial Evaluation}

\author{Julian Michael}
% add yourself here


\affil[ ]{CSE 599: Empirical Foundations of Machine Learning}
\affil[ ]{Paul G. Allen School of Computer Science \& Engineering, University of Washington}
\affil[ ]{\texttt{julianjm@cs.washington.edu}}

\date{}

\begin{document}

\maketitle

\begin{abstract}
  Adversarial evaluation and training have been proposed
\end{abstract}

\paragraph{Introduction}
End-to-end neural network models have had widespread success on standard benchmarks in NLP
\citep{wang-etal-2019-glue,wang-etal-2019-superglue,lee-etal-2018-end}.
However, models trained in the standard Empirical Risk Minimization paradigm (\eg, using
maximum-likelihood objectives) are liable to succeed in these settings by fitting to features or
correlations in the data which are ultimately not representative of the underlying task and fail to
generalize out of distribution, \eg, under domain shift or adversarial perturbation
\citep{gururangan-etal-2018-annotation,ilyas-etal-2019-adversarial}.
One promising method to overcome this difficulty is to move past the ERM paradigm and learn or
evaluate causal features which are invariant across domains or distributions of data.
While methods to do this often require the use of explicitly specified domains of data (cite DRO
stuff) \citep{peters-etal-2016-causal,arjovsky-etal-2020-invariant},
a more lightweight approach is adversarial evaluation and training
\citep{nie-etal-2020-adversarial,kiela-etal-2021-dynabench} which has annotators deliberately search
for examples on which a model fails.
Adversarial evaluations can help expose a model's shortcomings (cite example) and aid in training
more robust models.
However, the process of developing adversarial data is imperfect, and adversarial data may itself
not resemble naturalistic distributions.
For these reasons, it is not clear what a model's performance under adversarial evaluation implies
about its performance characteristics on naturalistic distributions,
and it is not clear in what ways training on adversarial data will aid a model's performance in
natural settings.

In this project, we propose to investigate the interplay of adversarial learning and evaluation with
\textit{annotation ambiguity}.
This is relevant because searching for adversarial examples may end up either oversampling
ambiguous/arguable examples that humans are likely to get `wrong', or, if the data is filtered for
high human agreement, it may systematically \textit{under}sample such cases.
In the former case, training on this data may have limited benefits, and in the latter case,
adversarial evaluation might underestimate model performance in realistic settings.

\paragraph{Research Question}
Do adversarial evaluation or training methods introduce systematic biases that ultimately
misrepresent a model's worst-case performance? In particular, do biases with respect to the
underlying ambiguity of the underlying data limit the benefits of the adversarial approach?

\paragraph{Methodology}

\paragraph{Data}

\paragraph{Implementation}

\paragraph{Impact}

\bibliographystyle{acl_natbib}
% \bibliography{references}
\bibliography{anthology,references}

\end{document}