
\usepackage{times}
\usepackage{latexsym}

\usepackage{url}
\renewcommand{\UrlFont}{\ttfamily\small}
\usepackage[utf8]{inputenc}
\usepackage{xcolor}

% Totally tabular!
\usepackage{booktabs}

\usepackage{graphicx}

\usepackage{caption}
\usepackage{subcaption}

\usepackage{amssymb}
\usepackage{amsmath}

\usepackage{textcomp}

\DeclareMathOperator*{\argmax}{argmax}
\DeclareMathOperator*{\argmin}{argmin}
\DeclareMathOperator*{\softmax}{softmax}
\DeclareMathOperator*{\logsumexp}{logsumexp}
\DeclareMathOperator*{\hardmax}{hardmax}
\DeclareMathOperator*{\entropy}{H}
\DeclareMathOperator{\expectation}{\mathbb{E}}
\DeclareMathOperator{\prob}{P}

\usepackage{hyperref}
\usepackage{cleveref}
\def\sectionautorefname{Section}
\def\subsectionautorefname{Section}

% (From ACL style:)
% This is not strictly necessary, and may be commented out,
% but it will improve the layout of the manuscript,
% and will typically save some space.
\usepackage{microtype}

% TODO: get a better draft watermark than this lol
% \usepackage{draftwatermark}
% \SetWatermarkText{DRAFT}
% \SetWatermarkScale{1}

% fix up author superscripts
\usepackage[blocks]{authblk}
\newcommand*\samethanks[1][\value{footnote}]{\footnotemark[#1]}
\renewcommand*{\Affilfont}{\normalsize\normalfont}
\renewcommand*{\Authfont}{\bfseries}
%% \renewcommand\footnotemark{}
%%\makeatletter
%%\renewcommand\AB@authnote[1]{\rlap{\textsuperscript{\normalfont#1}}}
%%\renewcommand\Authsep{,~~\,}
%%\renewcommand\Authand{~~and~}
%%\renewcommand\Authands{,~~and~}
%%\makeatother
% end fix up author superscripts

% appendix autoref patch (\autoref subsections in appendix)
% https://tex.stackexchange.com/questions/149807/autoref-subsections-in-appendix
\usepackage{appendix}
\usepackage{etoolbox}
\makeatletter
\patchcmd{\hyper@makecurrent}{%
    \ifx\Hy@param\Hy@chapterstring
        \let\Hy@param\Hy@chapapp
    \fi
}{%
    \iftoggle{inappendix}{%true-branch
        % list the names of all sectioning counters here
        \@checkappendixparam{chapter}%
        \@checkappendixparam{section}%
        \@checkappendixparam{subsection}%
        \@checkappendixparam{subsubsection}%
        \@checkappendixparam{paragraph}%
        \@checkappendixparam{subparagraph}%
    }{}%
}{}{\errmessage{failed to patch}}

\newcommand*{\@checkappendixparam}[1]{%
    \def\@checkappendixparamtmp{#1}%
    \ifx\Hy@param\@checkappendixparamtmp
        \let\Hy@param\Hy@appendixstring
    \fi
}
\makeatletter

\newtoggle{inappendix}
\togglefalse{inappendix}

\apptocmd{\appendix}{\toggletrue{inappendix}}{}{\errmessage{failed to patch}}
\apptocmd{\subappendices}{\toggletrue{inappendix}}{}{\errmessage{failed to patch}}
% end appendix autoref patch

% fancy underline
% https://alexwlchan.net/2017/10/latex-underlines/
\usepackage{contour}
\usepackage[normalem]{ulem}

\renewcommand{\ULdepth}{1.8pt}
\contourlength{0.8pt}

\newcommand{\nuline}[1]{%
  \uline{\phantom{#1}}%
  \llap{\contour{white}{#1}}%
}
% end fancy underline


\usepackage{todonotes}  % temporary - use this for margin notes in bubbles
\usepackage[normalem]{ulem}  % for \sout
\newif\ifcomments
\commentstrue
% \commentsfalse

\ifcomments
    \newcommand*{\TODO}[1]{\textcolor{red}{[TODO: #1]}}
    \newcommand*{\tocite}[1]{(\textcolor{blue}{#1})}
    \newcommand*{\tocitep}[1]{(\textcolor{blue}{#1})}
    \newcommand*{\tocitet}[1]{\textcolor{blue}{#1}}
    
    \newcommand*{\julian}[1]{\textcolor{orange}{[JJM: #1]}}
    \newcommand*{\ian}[1]{\textcolor{olive}{[IFT: #1]}}
    \newcommand*{\jan}[1]{\textcolor{purple}{[JAB: #1]}}
    
    \definecolor{iansnotecolor}{RGB}{255, 218, 181}
    \newcommand*{\iansidenote}[1]{
        \todo[color=iansnotecolor, size=\footnotesize]{%
        [\textbf{Ian:}] #1}%
    }
    
    \newcommand*{\proposed}[1]{\textcolor{purple}{#1}}
    \newcommand*{\maybedelete}[1]{\textcolor{red}{\sout{#1}}}
    \newcommand{\punch}[0]{\includegraphics[width=.04\textwidth]{images/falconpunch.png}}
\else
    %%
    % Disable user commands for submission or length check
    \newcommand*{\TODO}[1]{}
    \newcommand*{\tocite}[1]{}
    \newcommand*{\tocitep}[1]{}
    \newcommand*{\tocitet}[1]{}
    
    \newcommand*{\julian}[1]{}
    \newcommand*{\ian}[1]{}
    \newcommand*{\jan}[1]{}
    \newcommand*{\iansidenote}[1]{}
    
    \newcommand*{\maybedelete}[1]{}
    \newcommand*{\proposed}[1]{#1}
\fi